%% Generated by Sphinx.
%%% And edited by Łukasz Pawlak.
\ProvidesPackage{./sphinx_sty}

\def\sphinxdocclass{report}
\documentclass[letterpaper,10pt,english]{sphinxmanual}
\ifdefined\pdfpxdimen
   \let\sphinxpxdimen\pdfpxdimen\else\newdimen\sphinxpxdimen
\fi \sphinxpxdimen=.75bp\relax
\ifdefined\pdfimageresolution
    \pdfimageresolution= \numexpr \dimexpr1in\relax/\sphinxpxdimen\relax
\fi
%% let collapsible pdf bookmarks panel have high depth per default
\PassOptionsToPackage{bookmarksdepth=5}{hyperref}


\PassOptionsToPackage{warn}{textcomp}
\usepackage[utf8]{inputenc}
\ifdefined\DeclareUnicodeCharacter
% support both utf8 and utf8x syntaxes
  \ifdefined\DeclareUnicodeCharacterAsOptional
    \def\sphinxDUC#1{\DeclareUnicodeCharacter{"#1}}
  \else
    \let\sphinxDUC\DeclareUnicodeCharacter
  \fi
  \sphinxDUC{00A0}{\nobreakspace}
  \sphinxDUC{2500}{\sphinxunichar{2500}}
  \sphinxDUC{2502}{\sphinxunichar{2502}}
  \sphinxDUC{2514}{\sphinxunichar{2514}}
  \sphinxDUC{251C}{\sphinxunichar{251C}}
  \sphinxDUC{2572}{\textbackslash}
\fi
\usepackage{cmap}
\usepackage[T1]{fontenc}
\usepackage{amsmath,amssymb,amstext}
\usepackage{babel}
\usepackage{tgtermes}
\usepackage{tgheros}
\renewcommand{\ttdefault}{txtt}
\usepackage[Bjarne]{fncychap}
\usepackage{sphinx}
\fvset{fontsize=auto}
\usepackage{geometry}

% our blyatiful uml diagrams
\usepackage[simplified]{pgf-umlcd}
\usepackage[backend=biber,style=numeric]{biblatex} 
\addbibresource{sources.bib} 


% Include hyperref last.
\usepackage{hyperref}
% Fix anchor placement for figures with captions.
\usepackage{hypcap}% it must be loaded after hyperref.
% Set up styles of URL: it should be placed after hyperref.
\urlstyle{same}

\addto\captionsenglish{\renewcommand{\contentsname}{Contents:}}

\usepackage{sphinxmessages}
\setcounter{tocdepth}{1}



\title{todo\_project\_name}
\date{Jan 30, 2023}
\release{}
\author{Jakub Kaczor, Wojciech Michalczuk, Łukasz Pawlak}
\newcommand{\sphinxlogo}{\vbox{}}
\renewcommand{\releasename}{}
\makeindex
\begin{document}

\ifdefined\shorthandoff
  \ifnum\catcode`\=\string=\active\shorthandoff{=}\fi
  \ifnum\catcode`\"=\active\shorthandoff{"}\fi
\fi

\pagestyle{empty}
\sphinxmaketitle
\pagestyle{plain}
\sphinxtableofcontents
\pagestyle{normal}
\phantomsection\label{\detokenize{index::doc}}


\sphinxstepscope


\chapter{The general idea}
The purpose of this program is to allow the user to sign, and check RSA
signatures of files.

The standard use of this program is as follows:
Alice and Bob generate RSA keys. They exchange public keys with each other,
and keep private keys for themselves. Once Alice has Bob's public key, she
can verify Bob's signatures. If Bob wants to confirm that send message
indeed came from him, he can sign it with his private key, and send
signature alongside original file. Then Alice can verify if signature
matches acquired file, and Bob's public key.

It is extremely unlikely that another public key would match given
file and signature. It is hard to generate fake signature to
match fixed message and public key as well.
Thus, such signature can be a solid proof of identity
of sender, as long as his private key stays confidential.

The RSA is based around idea of ''factorization of large numbers is hard''
and modular arithmetic. Private key is pair $(d, n)$, where $d$ is
co-prime with $\phi(n)$, and $n$ is product of 2 large prime numbers $p, q$.
Public key is pair $(e, n)$, where $e \equiv d^{-1} \operatorname{mod} \phi(n)$.
To compute $d$ knowing only $e$ and $n = pq$, one would have to
compute $\varphi(pq) = (p - 1)(q - 1)$, for which factorization of $n$ is needed.
This is where security of RSA comes from.

To encrypt message $m$ (by message we mean natural number smaller than $n$;
regular data could be divided into blocks of appropriate size and interpreted
as numbers to fit this criteria) we compute $m' = m^{e}\operatorname{mod} n$.
To encrypt $m'$, we compute $(m')^{d}\operatorname{mod} n = m$. This works
thanks to identity $a^{ed}\equiv a \operatorname{mod} n$.

To sign file, we reverse this procedure in a way. First, Bob computes
hash of the message to sign $h(m)$. Hashed message should in general
be shorter, so that applying RSA is easier and less computationally expensive.
Then, Bob computes $h' = h(m)^d \operatorname{mod} n$.
To check signature, Alice computes $(h')^{e}\operatorname{mod}n$ using public key,
and compares it to $h(m)$.
This method allows anyone with access to file and public key to check Bob's
identity as the sender.

In our program, we use 2 hashing algorithms: MD4 and MD5. They are both
similar in structure. More about them can be found in papers \cite{md4}
and \cite{md5}

% \let\clearpage\relax
\printbibliography


\chapter{How to use the program}
The program offers 4 core functionalities:
\begin{enumerate}
    \item Hashing text files and other files, using MD4 or MD5 hash functions;
    \item Generating public-private RSA key pairs;
    \item Signing text message or file with RSA protocol;
    \item Veryfing RSA signature of text message or file.
\end{enumerate}
\section{Hashing}
To hash text message or other file do the following:
\begin{enumerate}
    \item Choose action {\em Generate checksum} from drop menu at the top of the window.
    \item Click {\em Change message path...} button and select file you want to hash.
    \item Choose hashing function (drop menu next to {\em algorithm} label).
    \item Choose checksum file destination and name by clicking {\em Change checksum path...} button.
    \item Click {\em Proceed} button.
\end{enumerate}
Hash of selected file should appear under path specified in step 4. This is text file containing
message digest encoded as hex string.
\section{Generating keys}
To generate RSA key pair do the following:
\begin{enumerate}
    \item Choose action {\em Generate key pair} from drop menu at the top of the window.
    \item Click {\em Change keypair save location...} and pick folder in which keys will be saved.
        Make sure it is private location, that other users can't access.
    \item Fill text field labelled {\em Base name of generated keys}. In the end, 2 files
        will be created in folder specified in step 2, with this base name.
    \item Fill text field labelled {\em Key pair id}. This field should contain description
        of whose key it is, or other identifier.
    \item Click {\em Proceed} button.
\end{enumerate}
Two files should appear in folder specified in step 2. They should have the same base name.
One should have extension \verb|.private|. This is the private key, which should not be shared.
The other should have extension \verb|.public|. This is the public key, and it should be shared
to make verification possible.
\section{Signing}
To sign text message or other file do the following:
\begin{enumerate}
    \item Choose action {\em Sign} from drop menu at the top of the window.
    \item Click {\em Change message path...} button and select file you want to sign.
    \item Click {\em Change private key...} button and select private key you want to sign with.
    \item Choose signature file destination and name by clicking {\em Change signature path...} button.
    \item Click {\em Proceed} button.
\end{enumerate}
Signature of selected file should appear under path specified in step 4.
This is text file containing number representation of signature.
\section{Veryfying signature}
To verify signature do the following:
\begin{enumerate}
    \item Choose action {\em Verify} from drop menu at the top of the window.
    \item Click {\em Change public key...} button and select public key that is supposed
        to match given signature.
    \item Click {\em Change message path...} button and select file whose signature you are checking.
    \item Choose signature file path by clicking {\em Change signature path...} button.
    \item Click {\em Proceed} button.
\end{enumerate}
Information, about whether the file signature is correct or not should appear in
info box. Be careful to pick the right key and signature file each time.

\chapter{Code structure}

% something goes wrong with picture placement
\vspace{1em}
Huge part of this program uses object oriented programming
as its paradigm. In such, a class diagrams are standard tools
of visualization. Below you can see diagrams for classes
RSAKey, its descendants and class RSAKeyPair.

On the next page you can find diagrams for classes MD*, which
are responsible for hashing algorithms.
\vspace{3em}

	\begin{tikzpicture}
		\begin{abstractclass}[text width = 5 cm]{ABC}{0, 2}
		\end{abstractclass}
		\begin{class}[text width = 12 cm]{RSAKey}{0 ,0}
			\inherit{ABC}
			\attribute{+ key : int}
			\attribute{+ modulus: int}
			\attribute{+ id: Optional[str] = None}
			\operation{+ \textunderscore\textunderscore init\textunderscore\textunderscore(self, key: int, modulus: int, id: Optional[str] = None) -\textgreater\ None}
			\operation{+ \textunderscore\textunderscore repr\textunderscore\textunderscore(self) -\textgreater\ str}
			\operation{+ \textunderscore\textunderscore eq\textunderscore\textunderscore(self, other: Type[RSAKey]) -\textgreater\ bool}
		\end{class}
	
		
		\begin{class}[text width = 5 cm]{RSAKeyPublic}{ -5 , -5}
			\inherit{RSAKey}
		\end{class}
		\begin{class}[text width = 5 cm]{RSAKeyPrivate}{5 , -5}
			\inherit{RSAKey}
		\end{class}
	
	
		\begin{class}[text width = 7 cm]{RSAKeyPair}{0 , -7}
			\attribute{+ private: Type[RSAKeyPrivate]}
			\attribute{+ public: Type[RSAKeyPublic]}	
		\end{class}
	
	
		\composition{RSAKeyPair}{}{}{RSAKeyPrivate}
		\composition{RSAKeyPair}{}{}{RSAKeyPublic}
	\end{tikzpicture}


\documentclass{article}

\usepackage[utf8]{inputenc}
\usepackage{polski}

\usepackage[simplified]{pgf-umlcd}
\begin{document}
    \begin{tikzpicture}
        \begin{abstractclass}[text width=12.5cm]{MDN}{0,10}
            % class constants
            \attribute{+ padding: bytes}
            \attribute{+ last32: int}
            \attribute{+ last64: int}
            % object attributes
            \attribute{/ digest: bytes}
%digest(self)
            \attribute{\# \textunderscore A: int}
            \attribute{\# \textunderscore B: int}
            \attribute{\# \textunderscore C: int}
            \attribute{\# \textunderscore D: int}

            % init
            \operation{+ \textunderscore\textunderscore init\textunderscore\textunderscore (self, message\textunderscore bytes: Iterator[bytes]) -\textgreater\ None }
            % abstract method
            \operation[0]{\# \textunderscore update(self, X: List[int]) -\textgreater None}

            % concrete methods
            \operation{+ string\textunderscore digest(self) -\textgreater str}
            \operation{\# \textunderscore run\textunderscore algorithm(self, message\textunderscore bytes: Iterator[bytes]) -\textgreater None}
            \operation{+ \underline{l\textunderscore roll}(X: int, s: int) -\textgreater int}

            \operation{+ \underline{from\textunderscore file}(cls, filename: str) -\textgreater MDN}
            \operation{+ \underline{from\textunderscore bytes}(cls, byte\textunderscore string: bytes) -\textgreater MDN}

            \operation{\# \underline{\textunderscore bytes\textunderscore as\textunderscore generator}(byte\textunderscore string: bytes) -\textgreater Iterator[bytes]}
            \operation{\# \underline{\textunderscore file\textunderscore bytes\textunderscore generator}(filename: str, *, page\textunderscore size: int = 4096) -\textgreater Iterator[bytes]}
        \end{abstractclass}

        \begin{class}[text width=6cm]{MD4}{4,0}
            \inherit{MDN}
            \attribute{+ ROUND\textunderscore2: int}
            \attribute{+ ROUND\textunderscore3: int}

            \operation{\# \textunderscore f(X, Y, Z)}
            \operation{\# \textunderscore g(X, Y, Z)}
            \operation{\# \textunderscore h(X, Y, Z)}
            \operation{\# \textunderscore round\textunderscore 1\textunderscore op(A, B, C, D, X, s)}
            \operation{\# \textunderscore round\textunderscore 2\textunderscore op(A, B, C, D, X, s)}
            \operation{\# \textunderscore round\textunderscore 3\textunderscore op(A, B, C, D, X, s)}

            \operation{\# \textunderscore update(self, X: List[int]) -\textgreater None}
        \end{class}
        \begin{class}[text width=6cm]{MD5}{-3,0}
            \inherit{MDN}
            \attribute{+ T: List[int]}

            \operation{\# \textunderscore f(X, Y, Z)}
            \operation{\# \textunderscore g(X, Y, Z)}
            \operation{\# \textunderscore h(X, Y, Z)}
            \operation{\# \textunderscore i(X, Y, Z)}
            \operation{\# \textunderscore round\textunderscore 1\textunderscore op(A, B, C, D, X, s, i)}
            \operation{\# \textunderscore round\textunderscore 2\textunderscore op(A, B, C, D, X, s, i)}
            \operation{\# \textunderscore round\textunderscore 3\textunderscore op(A, B, C, D, X, s, i)}
            \operation{\# \textunderscore round\textunderscore 4\textunderscore op(A, B, C, D, X, s, i)}
            \operation{\# \textunderscore update(self, X: List[int]) -\textgreater None}
        \end{class}
    \end{tikzpicture}
\end{document}


%% part generated by sphinx
%% (altered only a little bit)
\chapter{todo\_project\_name package}
\label{\detokenize{modules:todo-project-name}}\label{\detokenize{modules::doc}}
\sphinxstepscope


\section{todo\_project\_name.core module}
\label{\detokenize{todo_project_name:module-todo_project_name.core}}\label{\detokenize{todo_project_name:todo-project-name-core-module}}\index{module@\spxentry{module}!todo\_project\_name.core@\spxentry{todo\_project\_name.core}}\index{todo\_project\_name.core@\spxentry{todo\_project\_name.core}!module@\spxentry{module}}\index{md4\_string() (in module todo\_project\_name.core)@\spxentry{md4\_string()}\spxextra{in module todo\_project\_name.core}}

\begin{fulllineitems}
\phantomsection\label{\detokenize{todo_project_name:todo_project_name.core.md4_string}}
\pysigstartsignatures
\pysiglinewithargsret{\sphinxcode{\sphinxupquote{todo\_project\_name.core.}}\sphinxbfcode{\sphinxupquote{md4\_string}}}{\emph{\DUrole{n}{message}\DUrole{p}{:}\DUrole{w}{  }\DUrole{n}{str}}}{{ $\rightarrow$ str}}
\pysigstopsignatures
\sphinxAtStartPar
Returns md4 digest of given string encoded as UTF\sphinxhyphen{}8 byte strings.


\subsection{Parameters}
\label{\detokenize{todo_project_name:parameters}}
\sphinxAtStartPar
message
: string whose hash is to be computed.

\end{fulllineitems}

\index{md5\_string() (in module todo\_project\_name.core)@\spxentry{md5\_string()}\spxextra{in module todo\_project\_name.core}}

\begin{fulllineitems}
\phantomsection\label{\detokenize{todo_project_name:todo_project_name.core.md5_string}}
\pysigstartsignatures
\pysiglinewithargsret{\sphinxcode{\sphinxupquote{todo\_project\_name.core.}}\sphinxbfcode{\sphinxupquote{md5\_string}}}{\emph{\DUrole{n}{message}\DUrole{p}{:}\DUrole{w}{  }\DUrole{n}{str}}}{{ $\rightarrow$ str}}
\pysigstopsignatures
\sphinxAtStartPar
Returns md5 digest of given string encoded as UTF\sphinxhyphen{}8 byte strings.


\subsection{Parameters}
\label{\detokenize{todo_project_name:id1}}
\sphinxAtStartPar
message
: string whose hash is to be computed.

\end{fulllineitems}



\section{todo\_project\_name.find\_prime module}
\label{\detokenize{todo_project_name:module-todo_project_name.find_prime}}\label{\detokenize{todo_project_name:todo-project-name-find-prime-module}}\index{module@\spxentry{module}!todo\_project\_name.find\_prime@\spxentry{todo\_project\_name.find\_prime}}\index{todo\_project\_name.find\_prime@\spxentry{todo\_project\_name.find\_prime}!module@\spxentry{module}}\index{find\_prime() (in module todo\_project\_name.find\_prime)@\spxentry{find\_prime()}\spxextra{in module todo\_project\_name.find\_prime}}

\begin{fulllineitems}
\phantomsection\label{\detokenize{todo_project_name:todo_project_name.find_prime.find_prime}}
\pysigstartsignatures
\pysiglinewithargsret{\sphinxcode{\sphinxupquote{todo\_project\_name.find\_prime.}}\sphinxbfcode{\sphinxupquote{find\_prime}}}{\emph{\DUrole{n}{n}\DUrole{p}{:}\DUrole{w}{  }\DUrole{n}{int}}}{{ $\rightarrow$ int}}
\pysigstopsignatures
\sphinxAtStartPar
Return \sphinxtitleref{n}\sphinxhyphen{}bit probable prime.


\subsection{Parameters}
\label{\detokenize{todo_project_name:id2}}
\sphinxAtStartPar
\sphinxtitleref{n}
: number of bits, must be greater than 1,
\begin{quote}

\sphinxAtStartPar
because otherwise such a prime doesn’t exist.
\end{quote}

\end{fulllineitems}

\index{is\_probable\_prime() (in module todo\_project\_name.find\_prime)@\spxentry{is\_probable\_prime()}\spxextra{in module todo\_project\_name.find\_prime}}

\begin{fulllineitems}
\phantomsection\label{\detokenize{todo_project_name:todo_project_name.find_prime.is_probable_prime}}
\pysigstartsignatures
\pysiglinewithargsret{\sphinxcode{\sphinxupquote{todo\_project\_name.find\_prime.}}\sphinxbfcode{\sphinxupquote{is\_probable\_prime}}}{\emph{\DUrole{n}{candidate}\DUrole{p}{:}\DUrole{w}{  }\DUrole{n}{int}}}{{ $\rightarrow$ bool}}
\pysigstopsignatures
\sphinxAtStartPar
Check if \sphinxtitleref{candidate} is a probable prime.


\subsection{Notes}
\label{\detokenize{todo_project_name:notes}}
\sphinxAtStartPar
This function uses Rabin\sphinxhyphen{}Miller test under the hood.

\end{fulllineitems}



\section{todo\_project\_name.md4 module}
\label{\detokenize{todo_project_name:module-todo_project_name.md4}}\label{\detokenize{todo_project_name:todo-project-name-md4-module}}\index{module@\spxentry{module}!todo\_project\_name.md4@\spxentry{todo\_project\_name.md4}}\index{todo\_project\_name.md4@\spxentry{todo\_project\_name.md4}!module@\spxentry{module}}\index{MD4 (class in todo\_project\_name.md4)@\spxentry{MD4}\spxextra{class in todo\_project\_name.md4}}

\begin{fulllineitems}
\phantomsection\label{\detokenize{todo_project_name:todo_project_name.md4.MD4}}
\pysigstartsignatures
\pysiglinewithargsret{\sphinxbfcode{\sphinxupquote{class\DUrole{w}{  }}}\sphinxcode{\sphinxupquote{todo\_project\_name.md4.}}\sphinxbfcode{\sphinxupquote{MD4}}}{\emph{\DUrole{n}{message\_bytes}\DUrole{p}{:}\DUrole{w}{  }\DUrole{n}{Iterator\DUrole{p}{{[}}bytes\DUrole{p}{{]}}}}}{}
\pysigstopsignatures
\sphinxAtStartPar
Bases: {\hyperref[\detokenize{todo_project_name:todo_project_name.mdn.MDN}]{\sphinxcrossref{\sphinxcode{\sphinxupquote{MDN}}}}}

\sphinxAtStartPar
Class computing MD4 message digest. Works for little\sphinxhyphen{}endian architecture.

\sphinxAtStartPar
It is recommended to use methods \sphinxtitleref{MD4.from\_bytes} or \sphinxtitleref{MD4.from\_file}
to create new objects.

\sphinxAtStartPar
To get message digest as \sphinxtitleref{str} use \sphinxtitleref{string\_digest} method.
To get message digest as \sphinxtitleref{bytes} read \sphinxtitleref{digest} property.
\index{ROUND\_2 (todo\_project\_name.md4.MD4 attribute)@\spxentry{ROUND\_2}\spxextra{todo\_project\_name.md4.MD4 attribute}}

\begin{fulllineitems}
\phantomsection\label{\detokenize{todo_project_name:todo_project_name.md4.MD4.ROUND_2}}
\pysigstartsignatures
\pysigline{\sphinxbfcode{\sphinxupquote{ROUND\_2}}\sphinxbfcode{\sphinxupquote{\DUrole{w}{  }\DUrole{p}{=}\DUrole{w}{  }1518500249}}}
\pysigstopsignatures
\end{fulllineitems}

\index{ROUND\_3 (todo\_project\_name.md4.MD4 attribute)@\spxentry{ROUND\_3}\spxextra{todo\_project\_name.md4.MD4 attribute}}

\begin{fulllineitems}
\phantomsection\label{\detokenize{todo_project_name:todo_project_name.md4.MD4.ROUND_3}}
\pysigstartsignatures
\pysigline{\sphinxbfcode{\sphinxupquote{ROUND\_3}}\sphinxbfcode{\sphinxupquote{\DUrole{w}{  }\DUrole{p}{=}\DUrole{w}{  }1859775393}}}
\pysigstopsignatures
\end{fulllineitems}


\end{fulllineitems}



\section{todo\_project\_name.md5 module}
\label{\detokenize{todo_project_name:module-todo_project_name.md5}}\label{\detokenize{todo_project_name:todo-project-name-md5-module}}\index{module@\spxentry{module}!todo\_project\_name.md5@\spxentry{todo\_project\_name.md5}}\index{todo\_project\_name.md5@\spxentry{todo\_project\_name.md5}!module@\spxentry{module}}\index{MD5 (class in todo\_project\_name.md5)@\spxentry{MD5}\spxextra{class in todo\_project\_name.md5}}

\begin{fulllineitems}
\phantomsection\label{\detokenize{todo_project_name:todo_project_name.md5.MD5}}
\pysigstartsignatures
\pysiglinewithargsret{\sphinxbfcode{\sphinxupquote{class\DUrole{w}{  }}}\sphinxcode{\sphinxupquote{todo\_project\_name.md5.}}\sphinxbfcode{\sphinxupquote{MD5}}}{\emph{\DUrole{n}{message\_bytes}\DUrole{p}{:}\DUrole{w}{  }\DUrole{n}{Iterator\DUrole{p}{{[}}bytes\DUrole{p}{{]}}}}}{}
\pysigstopsignatures
\sphinxAtStartPar
Bases: {\hyperref[\detokenize{todo_project_name:todo_project_name.mdn.MDN}]{\sphinxcrossref{\sphinxcode{\sphinxupquote{MDN}}}}}

\sphinxAtStartPar
Class computing MD5 message digest. Works for little\sphinxhyphen{}endian architecture.

\sphinxAtStartPar
It is recommended to use methods \sphinxtitleref{MD5.from\_bytes} or \sphinxtitleref{MD5.from\_file}
to create new objects.

\sphinxAtStartPar
To get message digest as \sphinxtitleref{str} use \sphinxtitleref{string\_digest} method.
To get message digest as \sphinxtitleref{bytes} read \sphinxtitleref{digest} property.


\end{fulllineitems}



\section{todo\_project\_name.mdn module}
\label{\detokenize{todo_project_name:module-todo_project_name.mdn}}\label{\detokenize{todo_project_name:todo-project-name-mdn-module}}\index{module@\spxentry{module}!todo\_project\_name.mdn@\spxentry{todo\_project\_name.mdn}}\index{todo\_project\_name.mdn@\spxentry{todo\_project\_name.mdn}!module@\spxentry{module}}\index{MDN (class in todo\_project\_name.mdn)@\spxentry{MDN}\spxextra{class in todo\_project\_name.mdn}}

\begin{fulllineitems}
\phantomsection\label{\detokenize{todo_project_name:todo_project_name.mdn.MDN}}
\pysigstartsignatures
\pysiglinewithargsret{\sphinxbfcode{\sphinxupquote{class\DUrole{w}{  }}}\sphinxcode{\sphinxupquote{todo\_project\_name.mdn.}}\sphinxbfcode{\sphinxupquote{MDN}}}{\emph{\DUrole{n}{message\_bytes}\DUrole{p}{:}\DUrole{w}{  }\DUrole{n}{Iterator\DUrole{p}{{[}}bytes\DUrole{p}{{]}}}}}{}
\pysigstopsignatures
\sphinxAtStartPar
Bases: \sphinxcode{\sphinxupquote{ABC}}

\sphinxAtStartPar
Superclass of MD4 and MD5. Works for little\sphinxhyphen{}endian architecture.
\index{digest (todo\_project\_name.mdn.MDN property)@\spxentry{digest}\spxextra{todo\_project\_name.mdn.MDN property}}

\begin{fulllineitems}
\phantomsection\label{\detokenize{todo_project_name:todo_project_name.mdn.MDN.digest}}
\pysigstartsignatures
\pysigline{\sphinxbfcode{\sphinxupquote{property\DUrole{w}{  }}}\sphinxbfcode{\sphinxupquote{digest}}}
\pysigstopsignatures
\sphinxAtStartPar
The message digest as bytes.

\end{fulllineitems}

\index{from\_bytes() (todo\_project\_name.mdn.MDN class method)@\spxentry{from\_bytes()}\spxextra{todo\_project\_name.mdn.MDN class method}}

\begin{fulllineitems}
\phantomsection\label{\detokenize{todo_project_name:todo_project_name.mdn.MDN.from_bytes}}
\pysigstartsignatures
\pysiglinewithargsret{\sphinxbfcode{\sphinxupquote{classmethod\DUrole{w}{  }}}\sphinxbfcode{\sphinxupquote{from\_bytes}}}{\emph{\DUrole{n}{byte\_string}\DUrole{p}{:}\DUrole{w}{  }\DUrole{n}{bytes}}}{{ $\rightarrow$ {\hyperref[\detokenize{todo_project_name:todo_project_name.mdn.MDN}]{\sphinxcrossref{MDN}}}}}
\pysigstopsignatures
\sphinxAtStartPar
This function serves as constructor, which allows to compute hash
of \sphinxtitleref{bytes}.


\subsection{Parameters}
\label{\detokenize{todo_project_name:id3}}
\sphinxAtStartPar
byte\_string
: message whose digest is to be computed.

\end{fulllineitems}

\index{from\_file() (todo\_project\_name.mdn.MDN class method)@\spxentry{from\_file()}\spxextra{todo\_project\_name.mdn.MDN class method}}

\begin{fulllineitems}
\phantomsection\label{\detokenize{todo_project_name:todo_project_name.mdn.MDN.from_file}}
\pysigstartsignatures
\pysiglinewithargsret{\sphinxbfcode{\sphinxupquote{classmethod\DUrole{w}{  }}}\sphinxbfcode{\sphinxupquote{from\_file}}}{\emph{\DUrole{n}{filename}\DUrole{p}{:}\DUrole{w}{  }\DUrole{n}{str}}}{{ $\rightarrow$ {\hyperref[\detokenize{todo_project_name:todo_project_name.mdn.MDN}]{\sphinxcrossref{MDN}}}}}
\pysigstopsignatures
\sphinxAtStartPar
This function serves as constructor, which allows to compute hash
of file under given path.


\subsection{Parameters}
\label{\detokenize{todo_project_name:id4}}
\sphinxAtStartPar
filename
: path to existing file whose digest is to be computed.

\end{fulllineitems}

\index{l\_roll() (todo\_project\_name.mdn.MDN static method)@\spxentry{l\_roll()}\spxextra{todo\_project\_name.mdn.MDN static method}}

\begin{fulllineitems}
\phantomsection\label{\detokenize{todo_project_name:todo_project_name.mdn.MDN.l_roll}}
\pysigstartsignatures
\pysiglinewithargsret{\sphinxbfcode{\sphinxupquote{static\DUrole{w}{  }}}\sphinxbfcode{\sphinxupquote{l\_roll}}}{\emph{\DUrole{n}{X}\DUrole{p}{:}\DUrole{w}{  }\DUrole{n}{int}}, \emph{\DUrole{n}{s}\DUrole{p}{:}\DUrole{w}{  }\DUrole{n}{int}}}{{ $\rightarrow$ int}}
\pysigstopsignatures
\sphinxAtStartPar
Roll (rotate) bits of 32\sphinxhyphen{}bit unsigned integer \sphinxtitleref{s} positions
to the left.


\subsection{Parameters}
\label{\detokenize{todo_project_name:id5}}
\sphinxAtStartPar
X: integer to be rolled. Its binary representation cannot exceed 32 bits.

\sphinxAtStartPar
s: number of digits to roll. Must be integer in {[}0, 32{]}.

\end{fulllineitems}

\index{last32 (todo\_project\_name.mdn.MDN attribute)@\spxentry{last32}\spxextra{todo\_project\_name.mdn.MDN attribute}}

\begin{fulllineitems}
\phantomsection\label{\detokenize{todo_project_name:todo_project_name.mdn.MDN.last32}}
\pysigstartsignatures
\pysigline{\sphinxbfcode{\sphinxupquote{last32}}\sphinxbfcode{\sphinxupquote{\DUrole{w}{  }\DUrole{p}{=}\DUrole{w}{  }4294967295}}}
\pysigstopsignatures
\end{fulllineitems}

\index{last64 (todo\_project\_name.mdn.MDN attribute)@\spxentry{last64}\spxextra{todo\_project\_name.mdn.MDN attribute}}

\begin{fulllineitems}
\phantomsection\label{\detokenize{todo_project_name:todo_project_name.mdn.MDN.last64}}
\pysigstartsignatures
\pysigline{\sphinxbfcode{\sphinxupquote{last64}}\sphinxbfcode{\sphinxupquote{\DUrole{w}{  }\DUrole{p}{=}\DUrole{w}{  }18446744073709551615}}}
\pysigstopsignatures
\end{fulllineitems}

\index{padding (todo\_project\_name.mdn.MDN attribute)@\spxentry{padding}\spxextra{todo\_project\_name.mdn.MDN attribute}}


\index{string\_digest() (todo\_project\_name.mdn.MDN method)@\spxentry{string\_digest()}\spxextra{todo\_project\_name.mdn.MDN method}}

\begin{fulllineitems}
\phantomsection\label{\detokenize{todo_project_name:todo_project_name.mdn.MDN.string_digest}}
\pysigstartsignatures
\pysiglinewithargsret{\sphinxbfcode{\sphinxupquote{string\_digest}}}{}{{ $\rightarrow$ str}}
\pysigstopsignatures
\sphinxAtStartPar
Returns string representation of message digest.

\end{fulllineitems}


\end{fulllineitems}



\section{todo\_project\_name.rsa module}
\label{\detokenize{todo_project_name:module-todo_project_name.rsa}}\label{\detokenize{todo_project_name:todo-project-name-rsa-module}}\index{module@\spxentry{module}!todo\_project\_name.rsa@\spxentry{todo\_project\_name.rsa}}\index{todo\_project\_name.rsa@\spxentry{todo\_project\_name.rsa}!module@\spxentry{module}}\index{RSAKey (class in todo\_project\_name.rsa)@\spxentry{RSAKey}\spxextra{class in todo\_project\_name.rsa}}

\begin{fulllineitems}
\phantomsection\label{\detokenize{todo_project_name:todo_project_name.rsa.RSAKey}}
\pysigstartsignatures
\pysiglinewithargsret{\sphinxbfcode{\sphinxupquote{class\DUrole{w}{  }}}\sphinxcode{\sphinxupquote{todo\_project\_name.rsa.}}\sphinxbfcode{\sphinxupquote{RSAKey}}}{\emph{\DUrole{n}{key}\DUrole{p}{:}\DUrole{w}{  }\DUrole{n}{int}}, \emph{\DUrole{n}{modulus}\DUrole{p}{:}\DUrole{w}{  }\DUrole{n}{int}}, \emph{\DUrole{n}{id}\DUrole{p}{:}\DUrole{w}{  }\DUrole{n}{Optional\DUrole{p}{{[}}str\DUrole{p}{{]}}}\DUrole{w}{  }\DUrole{o}{=}\DUrole{w}{  }\DUrole{default_value}{None}}}{}
\pysigstopsignatures
\sphinxAtStartPar
Bases: \sphinxcode{\sphinxupquote{ABC}}

\end{fulllineitems}

\index{RSAKeyPair (class in todo\_project\_name.rsa)@\spxentry{RSAKeyPair}\spxextra{class in todo\_project\_name.rsa}}

\begin{fulllineitems}
\phantomsection\label{\detokenize{todo_project_name:todo_project_name.rsa.RSAKeyPair}}
\pysigstartsignatures
\pysiglinewithargsret{\sphinxbfcode{\sphinxupquote{class\DUrole{w}{  }}}\sphinxcode{\sphinxupquote{todo\_project\_name.rsa.}}\sphinxbfcode{\sphinxupquote{RSAKeyPair}}}{\emph{\DUrole{n}{public}\DUrole{p}{:}\DUrole{w}{  }\DUrole{n}{{\hyperref[\detokenize{todo_project_name:todo_project_name.rsa.RSAKeyPublic}]{\sphinxcrossref{todo\_project\_name.rsa.RSAKeyPublic}}}}}, \emph{\DUrole{n}{private}\DUrole{p}{:}\DUrole{w}{  }\DUrole{n}{{\hyperref[\detokenize{todo_project_name:todo_project_name.rsa.RSAKeyPrivate}]{\sphinxcrossref{todo\_project\_name.rsa.RSAKeyPrivate}}}}}}{}
\pysigstopsignatures
\sphinxAtStartPar
Bases: \sphinxcode{\sphinxupquote{object}}
\index{private (todo\_project\_name.rsa.RSAKeyPair attribute)@\spxentry{private}\spxextra{todo\_project\_name.rsa.RSAKeyPair attribute}}

\begin{fulllineitems}
\phantomsection\label{\detokenize{todo_project_name:todo_project_name.rsa.RSAKeyPair.private}}
\pysigstartsignatures
\pysigline{\sphinxbfcode{\sphinxupquote{private}}\sphinxbfcode{\sphinxupquote{\DUrole{p}{:}\DUrole{w}{  }{\hyperref[\detokenize{todo_project_name:todo_project_name.rsa.RSAKeyPrivate}]{\sphinxcrossref{RSAKeyPrivate}}}}}}
\pysigstopsignatures
\end{fulllineitems}

\index{public (todo\_project\_name.rsa.RSAKeyPair attribute)@\spxentry{public}\spxextra{todo\_project\_name.rsa.RSAKeyPair attribute}}

\begin{fulllineitems}
\phantomsection\label{\detokenize{todo_project_name:todo_project_name.rsa.RSAKeyPair.public}}
\pysigstartsignatures
\pysigline{\sphinxbfcode{\sphinxupquote{public}}\sphinxbfcode{\sphinxupquote{\DUrole{p}{:}\DUrole{w}{  }{\hyperref[\detokenize{todo_project_name:todo_project_name.rsa.RSAKeyPublic}]{\sphinxcrossref{RSAKeyPublic}}}}}}
\pysigstopsignatures
\end{fulllineitems}


\end{fulllineitems}

\index{RSAKeyPrivate (class in todo\_project\_name.rsa)@\spxentry{RSAKeyPrivate}\spxextra{class in todo\_project\_name.rsa}}

\begin{fulllineitems}
\phantomsection\label{\detokenize{todo_project_name:todo_project_name.rsa.RSAKeyPrivate}}
\pysigstartsignatures
\pysiglinewithargsret{\sphinxbfcode{\sphinxupquote{class\DUrole{w}{  }}}\sphinxcode{\sphinxupquote{todo\_project\_name.rsa.}}\sphinxbfcode{\sphinxupquote{RSAKeyPrivate}}}{\emph{\DUrole{n}{key}\DUrole{p}{:}\DUrole{w}{  }\DUrole{n}{int}}, \emph{\DUrole{n}{modulus}\DUrole{p}{:}\DUrole{w}{  }\DUrole{n}{int}}, \emph{\DUrole{n}{id}\DUrole{p}{:}\DUrole{w}{  }\DUrole{n}{Optional\DUrole{p}{{[}}str\DUrole{p}{{]}}}\DUrole{w}{  }\DUrole{o}{=}\DUrole{w}{  }\DUrole{default_value}{None}}}{}
\pysigstopsignatures
\sphinxAtStartPar
Bases: {\hyperref[\detokenize{todo_project_name:todo_project_name.rsa.RSAKey}]{\sphinxcrossref{\sphinxcode{\sphinxupquote{RSAKey}}}}}

\end{fulllineitems}

\index{RSAKeyPublic (class in todo\_project\_name.rsa)@\spxentry{RSAKeyPublic}\spxextra{class in todo\_project\_name.rsa}}

\begin{fulllineitems}
\phantomsection\label{\detokenize{todo_project_name:todo_project_name.rsa.RSAKeyPublic}}
\pysigstartsignatures
\pysiglinewithargsret{\sphinxbfcode{\sphinxupquote{class\DUrole{w}{  }}}\sphinxcode{\sphinxupquote{todo\_project\_name.rsa.}}\sphinxbfcode{\sphinxupquote{RSAKeyPublic}}}{\emph{\DUrole{n}{key}\DUrole{p}{:}\DUrole{w}{  }\DUrole{n}{int}}, \emph{\DUrole{n}{modulus}\DUrole{p}{:}\DUrole{w}{  }\DUrole{n}{int}}, \emph{\DUrole{n}{id}\DUrole{p}{:}\DUrole{w}{  }\DUrole{n}{Optional\DUrole{p}{{[}}str\DUrole{p}{{]}}}\DUrole{w}{  }\DUrole{o}{=}\DUrole{w}{  }\DUrole{default_value}{None}}}{}
\pysigstopsignatures
\sphinxAtStartPar
Bases: {\hyperref[\detokenize{todo_project_name:todo_project_name.rsa.RSAKey}]{\sphinxcrossref{\sphinxcode{\sphinxupquote{RSAKey}}}}}

\end{fulllineitems}

\index{read\_key() (in module todo\_project\_name.rsa)@\spxentry{read\_key()}\spxextra{in module todo\_project\_name.rsa}}

\begin{fulllineitems}
\phantomsection\label{\detokenize{todo_project_name:todo_project_name.rsa.read_key}}
\pysigstartsignatures
\pysiglinewithargsret{\sphinxcode{\sphinxupquote{todo\_project\_name.rsa.}}\sphinxbfcode{\sphinxupquote{read\_key}}}{\emph{\DUrole{n}{path}\DUrole{p}{:}\DUrole{w}{  }\DUrole{n}{Path}}, \emph{\DUrole{n}{key\_type}\DUrole{p}{:}\DUrole{w}{  }\DUrole{n}{Type\DUrole{p}{{[}}RSAKeyVar\DUrole{p}{{]}}}}}{{ $\rightarrow$ RSAKeyVar}}
\pysigstopsignatures
\sphinxAtStartPar
Read RSA key from the file.

\end{fulllineitems}

\index{rsa\_key\_gen() (in module todo\_project\_name.rsa)@\spxentry{rsa\_key\_gen()}\spxextra{in module todo\_project\_name.rsa}}

\begin{fulllineitems}
\phantomsection\label{\detokenize{todo_project_name:todo_project_name.rsa.rsa_key_gen}}
\pysigstartsignatures
\pysiglinewithargsret{\sphinxcode{\sphinxupquote{todo\_project\_name.rsa.}}\sphinxbfcode{\sphinxupquote{rsa\_key\_gen}}}{\emph{\DUrole{n}{N}\DUrole{p}{:}\DUrole{w}{  }\DUrole{n}{int}}}{{ $\rightarrow$ {\hyperref[\detokenize{todo_project_name:todo_project_name.rsa.RSAKeyPair}]{\sphinxcrossref{RSAKeyPair}}}}}
\pysigstopsignatures
\sphinxAtStartPar
Generate RSA key pair.

\sphinxAtStartPar
Takes number \sphinxtitleref{N} and returns RSAKeyPair with (2 * N)\sphinxhyphen{}bit modulus.


\subsection{Parameters}
\label{\detokenize{todo_project_name:id6}}
\sphinxAtStartPar
\sphinxtitleref{N}
: determines the strength of the protocol.

\end{fulllineitems}

\index{rsa\_sign() (in module todo\_project\_name.rsa)@\spxentry{rsa\_sign()}\spxextra{in module todo\_project\_name.rsa}}

\begin{fulllineitems}
\phantomsection\label{\detokenize{todo_project_name:todo_project_name.rsa.rsa_sign}}
\pysigstartsignatures
\pysiglinewithargsret{\sphinxcode{\sphinxupquote{todo\_project\_name.rsa.}}\sphinxbfcode{\sphinxupquote{rsa\_sign}}}{\emph{\DUrole{n}{message: str}}, \emph{\DUrole{n}{key: \textasciitilde{}todo\_project\_name.rsa.RSAKeyPrivate}}, \emph{\DUrole{n}{algorithm: \textasciitilde{}typing.Type{[}\textasciitilde{}typing.Union{[}\textasciitilde{}todo\_project\_name.md4.MD4}}, \emph{\DUrole{n}{\textasciitilde{}todo\_project\_name.md5.MD5{]}{]} = \textless{}class \textquotesingle{}todo\_project\_name.md4.MD4\textquotesingle{}\textgreater{}}}}{{ $\rightarrow$ str}}
\pysigstopsignatures
\sphinxAtStartPar
Function returns a digital singnature based on the RSA protocol.


\subsection{Parameters}
\label{\detokenize{todo_project_name:id7}}
\sphinxAtStartPar
message
: string message to be singed

\sphinxAtStartPar
key
: RSA private key

\sphinxAtStartPar
algorithm
: hash method. Default: MD4.
Available algorithms: MD4, MD5.

\end{fulllineitems}

\index{rsa\_sign\_file() (in module todo\_project\_name.rsa)@\spxentry{rsa\_sign\_file()}\spxextra{in module todo\_project\_name.rsa}}

\begin{fulllineitems}
\phantomsection\label{\detokenize{todo_project_name:todo_project_name.rsa.rsa_sign_file}}
\pysigstartsignatures
\pysiglinewithargsret{\sphinxcode{\sphinxupquote{todo\_project\_name.rsa.}}\sphinxbfcode{\sphinxupquote{rsa\_sign\_file}}}{\emph{\DUrole{n}{filename: str}}, \emph{\DUrole{n}{key: \textasciitilde{}todo\_project\_name.rsa.RSAKeyPrivate}}, \emph{\DUrole{n}{algorithm: \textasciitilde{}typing.Type{[}\textasciitilde{}typing.Union{[}\textasciitilde{}todo\_project\_name.md4.MD4}}, \emph{\DUrole{n}{\textasciitilde{}todo\_project\_name.md5.MD5{]}{]} = \textless{}class \textquotesingle{}todo\_project\_name.md4.MD4\textquotesingle{}\textgreater{}}}}{{ $\rightarrow$ str}}
\pysigstopsignatures
\sphinxAtStartPar
Function returns a digital singnature based on the RSA protocol.


\subsection{Parameters}
\label{\detokenize{todo_project_name:id8}}
\sphinxAtStartPar
filename
: path to existing file to sign

\sphinxAtStartPar
key
: RSA private key

\sphinxAtStartPar
algorithm
: hash method. Default: MD4.
Available algorithms: MD4, MD5.

\end{fulllineitems}

\index{rsa\_verify() (in module todo\_project\_name.rsa)@\spxentry{rsa\_verify()}\spxextra{in module todo\_project\_name.rsa}}

\begin{fulllineitems}
\phantomsection\label{\detokenize{todo_project_name:todo_project_name.rsa.rsa_verify}}
\pysigstartsignatures
\pysiglinewithargsret{\sphinxcode{\sphinxupquote{todo\_project\_name.rsa.}}\sphinxbfcode{\sphinxupquote{rsa\_verify}}}{\emph{\DUrole{n}{message: str}}, \emph{\DUrole{n}{signature: str}}, \emph{\DUrole{n}{key: \textasciitilde{}todo\_project\_name.rsa.RSAKeyPublic}}, \emph{\DUrole{n}{algorithm: \textasciitilde{}typing.Type{[}\textasciitilde{}typing.Union{[}\textasciitilde{}todo\_project\_name.md4.MD4}}, \emph{\DUrole{n}{\textasciitilde{}todo\_project\_name.md5.MD5{]}{]} = \textless{}class \textquotesingle{}todo\_project\_name.md4.MD4\textquotesingle{}\textgreater{}}}}{}
\pysigstopsignatures
\sphinxAtStartPar
Function verifies digital singnature of a message basing on the RSA protocol.
It compares decoded signature with hashed message
and returns True if they are the same, otherwise False.


\subsection{Parameters}
\label{\detokenize{todo_project_name:id9}}
\sphinxAtStartPar
message
: string message

\sphinxAtStartPar
signature
: signature for verification

\sphinxAtStartPar
key
: RSA public key

\sphinxAtStartPar
algorithm
: hash algorithm. Default: MD4.
Available algorithms: MD4, MD5.

\end{fulllineitems}

\index{rsa\_verify\_file() (in module todo\_project\_name.rsa)@\spxentry{rsa\_verify\_file()}\spxextra{in module todo\_project\_name.rsa}}

\begin{fulllineitems}
\phantomsection\label{\detokenize{todo_project_name:todo_project_name.rsa.rsa_verify_file}}
\pysigstartsignatures
\pysiglinewithargsret{\sphinxcode{\sphinxupquote{todo\_project\_name.rsa.}}\sphinxbfcode{\sphinxupquote{rsa\_verify\_file}}}{\emph{\DUrole{n}{filename: str}}, \emph{\DUrole{n}{signature: str}}, \emph{\DUrole{n}{key: \textasciitilde{}todo\_project\_name.rsa.RSAKeyPublic}}, \emph{\DUrole{n}{algorithm: \textasciitilde{}typing.Type{[}\textasciitilde{}typing.Union{[}\textasciitilde{}todo\_project\_name.md4.MD4}}, \emph{\DUrole{n}{\textasciitilde{}todo\_project\_name.md5.MD5{]}{]} = \textless{}class \textquotesingle{}todo\_project\_name.md4.MD4\textquotesingle{}\textgreater{}}}}{}
\pysigstopsignatures
\sphinxAtStartPar
Function verifies digital singnature of a message basing on the RSA protocol.
It compares decoded signature with hashed message
and returns True if they are the same, otherwise False.


\subsection{Parameters}
\label{\detokenize{todo_project_name:id10}}
\sphinxAtStartPar
filename
: path to file against which signature is being checked

\sphinxAtStartPar
signature
: signature for verification

\sphinxAtStartPar
key
: RSA public key

\sphinxAtStartPar
algorithm
: hash algorithm. Default: MD4.
Available algorithms: MD4, MD5.

\end{fulllineitems}

\index{save\_key() (in module todo\_project\_name.rsa)@\spxentry{save\_key()}\spxextra{in module todo\_project\_name.rsa}}

\begin{fulllineitems}
\phantomsection\label{\detokenize{todo_project_name:todo_project_name.rsa.save_key}}
\pysigstartsignatures
\pysiglinewithargsret{\sphinxcode{\sphinxupquote{todo\_project\_name.rsa.}}\sphinxbfcode{\sphinxupquote{save\_key}}}{\emph{\DUrole{n}{key}\DUrole{p}{:}\DUrole{w}{  }\DUrole{n}{{\hyperref[\detokenize{todo_project_name:todo_project_name.rsa.RSAKey}]{\sphinxcrossref{RSAKey}}}}}, \emph{\DUrole{n}{path}\DUrole{p}{:}\DUrole{w}{  }\DUrole{n}{Path}}}{{ $\rightarrow$ Path}}
\pysigstopsignatures
\sphinxAtStartPar
Save RSA key to the file.

\end{fulllineitems}

\end{document}
